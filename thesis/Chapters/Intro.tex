\chapter{Zielstellung und Aufbau der Arbeit}
\lhead{Kapitel 1: \emph{Zielstellung und Aufbau der Arbeit}}

In der Lichttechnik haben sich je nach Anwendungsbereich verschiedene Lichtsteuerprotokolle durchgesetzt. In dieser Bachelorarbeit werden verschiedene Lichtsteuerprotokolle miteinander verglichen und die spezifischen Anwendungsgebiete herausgearbeitet.

Für die Lehre wird ein DMX-Analysator gebaut. Diese Bachelorarbeit setzt daher den Schwerpunkt auf das DMX Protokoll. Dieser Analysator funktioniert eigenständig und benötigt keine externe Hard- oder Software. Dieses Projekt wird eingesetzt, um Studierenden das DMX Protokoll zu erklären und die Stärken und schwächen des Protokolls zu zeigen.

Das DMX Protokoll ist in der Bühnentechnik ein etabliertes Protokoll. Es gibt bereits einige gute Internetseiten, die das DMX Protokoll aus der Anwendersicht verständlich beschreiben. In dieser Bachelorarbeit wird jedoch der Fokus nicht nur auf die Anwenderschicht gesetzt, sondern auch auf die genaue Funktionsweise des Protokolls. Dabei wird beschrieben, wie die Daten im Kabel übertragen und die Informationen decodiert werden.